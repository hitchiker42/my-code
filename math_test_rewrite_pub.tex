% Created 2014-10-07 Tue 17:40
\documentclass[11pt]{article}
\usepackage[utf8]{inputenc}
\usepackage[T1]{fontenc}
\usepackage{fixltx2e}
\usepackage{graphicx}
\usepackage{longtable}
\usepackage{float}
\usepackage{wrapfig}
\usepackage{rotating}
\usepackage[normalem]{ulem}
\usepackage{amsmath}
\usepackage{textcomp}
\usepackage{marvosym}
\usepackage{wasysym}
\usepackage{amssymb}
\usepackage{hyperref}
\tolerance=1000
\usepackage{fullpage}
\author{Tucker DiNapoli}
\date{\today}
\title{Complex Analysis Test Redo}
\hypersetup{
  pdfkeywords={},
  pdfsubject={},
  pdfcreator={Emacs 24.4.50.1 (Org mode 8.2.6)}}
\begin{document}

\maketitle
\begin{enumerate}
\item Write as a+ib;\\
\begin{enumerate}
  \item \(z = \overline{i(4+4i-1)} = \overline{4i-4+i} = \overline{3i-4} = -4-3i\)\\
  \item \(z = (1-\sqrt{3}i)^{3} = (1-2\sqrt{3}i-3)(1-\sqrt{3}i) =
  1-\sqrt{3}i-2\sqrt{3}i + 2*3i-3 + 3\sqrt{3}i = -2+6i\)
\end{enumerate}

\item solve: \(x^{\text{4}}\) + \(6x^{\text{2}}\) + 3 = 0\\
  let y = \(x^{\text{2}}\), then \(y^{\text{2}}\) +6y +3 = 0
complete the square to obtain \(y^{\text{2}}\)+2*3y+9-9+3=0 factoring this gives
(y+3) = 6. this equation has 2 roots in terms of y:
\(y = -3 \pm \sqrt{6}.\) Recalling that y is really x each of these roots is really
two roots in terms of x giving us a total of 4 roots, which are:\\
\(x = +\sqrt\{-3+\sqrt{6}\}\\
  x = +\sqrt\{-3-\sqrt{6}\}\\
  x = -\sqrt\{-3+\sqrt{6}\}\\
  x = -\sqrt\{-3-\sqrt{6}\}\\\)

\item plot x,\(\bar{z}\),-z,-\(\bar{z}\)\\
  compute the modulus and argument of z for \(z = \sqrt{2}/2 -i\sqrt{2}/2\)
\(z = \sqrt{2}/2 -i\sqrt{2}/2\), rewritting this in polar form we obtain
\(z = cos(\sqrt{2}/2) -isin(\sqrt{2}/2) or z = /4 -i/4\)
rewritting this in exponential notation for convience we get z = e =
e knowing this we can eaisly write \(-z,\overline{z},-\overline{z}\) as:
-z = -e = e = e
\(\overline{z}\) = e
\(-\overline{z}\) = -e = e = e

Using exponential notation it is easy to see the modulus and argument of the
four points. For all 4 points the modulus is 1 while the arguments are /4,
3/4, 5/4 and 7/4 for \(\bar{z}\), -z, -\(\bar{z}\) and z respectively.

\item compute \(|\frac{(1+i)^{5}}{(-2+2i)^{6}}|\)\\
  \(|\frac{(1+i)^5}{(-2+2i)^6}|\)
In order to simplify the equation we re-write (-2+2i):
(-2+2i) = (2+2i) = (2*(1+i)) = 2*(1+i)
so we can rewrite the initial equation as: \(|\frac{1}{2^6(1+i)}|\)
this simplifies to (2+2), taking the modulus of this we obtain
\(\sqrt\{(2+2i)*(2-2i)\} = \sqrt{(2*2^6)^-1} = 2\)

5.plot the points satisfying:\\
\begin{enumerate}
  \item |z+2+i| = 1
 if we let z` = z+2+i then |z`| = 1 and the plot for z` is the
  unit circle. This implies that the plot of z will be a circle of radius 1
  centered at some unknown point.
  if we rewrite z as x+iy we obtain |x + 2 +i + iy| or |(x+2) + i(y+1)| = 1
  expanding the modulus and squaring both sizes we get (x+2) + (y+1) =1, we can
  then see that the plot is a circle centered at (-2,-1)

\item |z+1| + |z-1| = 4
As with the last problem we can rewrite z as x + iy and obtain
|(x+1) + iy| + |(x-1) + iy| = 4
expanding the left side we get \(\sqrt{(x+1)^2 + y^2} + \sqrt{(x-1)^2 +y^2} = 4\)
\end{enumerate}
\item use the triangle inequality to show: \(z-1\leq2 for |z|\leq1\)\\
  the triangle inequality states that for a triangle with sides x y z and z, then
\(z \leq x + y\). this implies that for vectors x and y, \(|x|-|y|\leq |x+y| \leq |x| + |y|\),
in this case z = |x + y|.

\item draw the complex number
z = 3(cos(-\(\frac{72\pi}{11}\)) + isin(\(-\frac{72\pi}{11}\)));

\item represent in polar form:
a) \(z = -1 - \sqrt{3}i\\
     z = x + iy; x = -1, y = \sqrt{3};\\
     z = r(cos(\theta)+ isin(\theta)); r = \sqrt{x^2+y^2} = 2\\
     sin(\theta) = \sqrt{3}/2; cos(\theta) = 1/2;\)\\
b) z = \(\frac{1+i}{1-i}\)

\item find the real and imaginary parts:\\
  a)z=(\(\frac{1-i}{1+i}\))
(1-i)/(1+i) is equal to -i.
by multiplying (1-i)/(1+i) by (1-i)/(1-i) we get (1-i)/2
(1-i) = 1 -2i -1 so (1-i)/2 = -2i/2 = -i
thus the problem is equivalent to the much simpler problem
z = (-i); the answer to which is that z = -1, so Re(z) = -1 and Im(z) = 0

\begin{enumerate}
\item z = \((-\sqrt{3}+i)^3\) \\
\end{enumerate}
let c = \(\sqrt{3}\)\\
  \(z = (-c+i)^3 = (-c+i)^2(-c+i) = c^2-2ic-1(-c+i) = -c^3+2ic^2+c+ic^2+2c-i
    = -c^3+3ic^2+3c-i = -3\sqrt{3}+9i+3\sqrt{3} -i = 9i-i = 8i\)
so z = 8i; Re(z)=0; Im(z) = 8

\item solve z+2iz+2-i=0 and represent the roots in the form a+ib\\

complete the square to obtain
(z+i) + (3-i) = 0
so \(z+i = \textpm{}\sqrt{-3+i}\) and \(z = -i \textpm{} \sqrt{-3+i}\)

expressed as a+ib the solutions are:
\(z = Re(\sqrt{-3+i}) + Im(\sqrt{-3+i})*i\) and
\(z = Re(-\sqrt{-3+i}) + Im(-\sqrt{-3+i})*i\)
\end{enumerate}
% Emacs 24.4.50.1 (Org mode 8.2.6)
\end{document}