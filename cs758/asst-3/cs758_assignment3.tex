% Created 2014-09-22 Mon 09:53
\documentclass[11pt]{article}
\usepackage[utf8]{inputenc}
\usepackage[T1]{fontenc}
\usepackage{fixltx2e}
\usepackage{graphicx}
\usepackage{longtable}
\usepackage{float}
\usepackage{wrapfig}
\usepackage{rotating}
\usepackage[normalem]{ulem}
\usepackage{amsmath}
\usepackage{textcomp}
\usepackage{marvosym}
\usepackage{wasysym}
\usepackage{amssymb}
\usepackage{hyperref}
\tolerance=1000
\author{Tucker DiNapoli}
\date{\today}
\title{cs758\_assignment3}
\hypersetup{
  pdfkeywords={},
  pdfsubject={},
  pdfcreator={Emacs 24.4.50.1 (Org mode 8.2.6)}}
\begin{document}

\maketitle
\tableofcontents

11.1-4: written in C like psuedo code
void insert (dictionary dict, element x)\{
  dict.arr[x.key] = x;
  mark$_{\text{used}}$(dict, x.key);
\}
void delete (dictionary dict, element x)\{
  dict.arr[x.key] = NIL;
  mark$_{\text{unused}}$(dict, x.key);
\}
element search (dictionary dict, key k)\{
  if (check$_{\text{used}}$(dict, k))\{
    dict.arr[k];
  \} else \{
    NIL;
  \}
\}


void mark$_{\text{used}}$(dictionary dict, key k)\{
  int x = dict.aux[k/sizeof(int)];
  x |= 1 << k \% sizeof(int);
\}
void mark$_{\text{unused}}$(dictionary dict, key k)\{
  int x = dict.aux[k/sizeof(int)];
  x \&=  \textasciitilde{}(1 << k \% sizeof(int));
\}
bool check$_{\text{used}}$(dictionary dict, key k)\{
  int x = dict.aux[k/sizeof(int)];
  return x \& (1 << k \% sizeof(int));
\}


void mark$_{\text{used}}$(dictionary dict, key k)\{
  push(dict.aux, k);
\}
void mark$_{\text{used}}$(dictionary dict, key k)\{
  for(i=0;i<dict.aux.len;i++)\{
    if(dict.aux[i] \texttt{= k)\{
      dict.aux[i] =} dict.aux[dict.aux.len--];
      return;
    \}
   \}
\}
bool check$_{\text{used}}$(dictionary dict, key k)\{
  for(i=0;i<dict.aux.len;i++)\{
    if(dict.aux[i] == k)\{
     return 1;
    \}
  \}
  return 0;
\}
void push(aux$_{\text{array}}$ arr, key k)\{
  if(arr.size \texttt{= arr.len)\{
    arr.size *} 2
    realloc(arr, arr.size);
  \}
  arr[arr.len++] = k
\}


11.2-6 (optional):

11.3-1:
search (list l, key k)
  hv = h(k)
  while(l.next)
   if(hv \texttt{= l.data.hv)
    if(k =} l.data.key)
      return l.data
    end
   end
   l = l.next
  end
  return nil
end
i.e store the hash value with the key and compare the hash values first, only
comparing the keys if the hash values match. Personally this is how I search
the buckets of a hash table using chaning.

12.1-2:
The difference between a binary search tree and a min heap is that a binary
search tree has an ordering between each element, whereas a min heap has an
ordering only between a parent node and it's children. In a binary tree for a
node n and it's children l and r the relation l <= n <= r holds, whereas in a
heap only n <= l and n <= r hold, there is no ordering beteween the children.

It is fairly easy to show that it takes O(n lg n) time to print out the keys in
a min heap of size n. Given that it takes O(n) time to print out n keys from a
sorted array and it takes time O(lg n) to remove the minimum element of a heap
(since each time the minimum element is removed the heap propertity
must be reestablished, which takes time O(lg n)) thus to print out the elements
from a heap of n elements in order will take O(n lg n)

12.2-4:
I've spent a while thinking about this, and I must misunderstand the question
because as far as I can tell this property is ensured by the ordering of a
binary tree. For a node b with a left child a and right child,  
a (and it's children) <= b <= c (and it's children) by defination.

I can't think of a way for this not to be true.
12.2-5(optional):
12.3-3:
we need to call tree-insert once for each element, ignoring time spent by
tree-insert this takes time O(N). Thus the best and worst behavior are going to
be the best and worst behavior of tree-insert * O(N).

The time complexity of tree-insert is O(h) where h is the height of the
tree. In the best case the height is lg(n) giving us a best case running time
of O(n lg n). In the worst case the height is n, giving a worst case running
time of O(n$^{\text{2}}$).
% Emacs 24.4.50.1 (Org mode 8.2.6)
\end{document}