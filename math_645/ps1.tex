% Created 2016-09-28 Wed 15:10
\documentclass[11pt]{article}
\usepackage[T1]{fontenc}
\usepackage{graphicx}
\usepackage{grffile}
\usepackage{longtable}
\usepackage{wrapfig}
\usepackage{rotating}
\usepackage[normalem]{ulem}
\usepackage{amsmath}
\usepackage{textcomp}
\usepackage{amssymb}
\usepackage{capt-of}
\usepackage{hyperref}
\usepackage{fontspec}
\usepackage{fullpage,fontspec,parskip}
\setmainfont[Ligatures=TeX]{Linux Libertine O}
\author{Tucker DiNapoli}
\date{\textit{<2016-09-28 Wed>}}
\title{Math 645 Problem Set 1}
\hypersetup{
 pdfauthor={Tucker DiNapoli},
 pdftitle={Math 645 Problem Set 1},
 pdfkeywords={},
 pdfsubject={},
 pdfcreator={Emacs 25.1.50.1 (Org mode 8.3.6)}, 
 pdflang={English}}
\begin{document}

\maketitle
\begin{enumerate}
\item §1.1 \#33,34
\begin{center}
\begin{tabular}{rrrr}
 & 20 & 20 & \\
10 & T1 & T2 & 40\\
10 & T4 & T3 & 40\\
 & 30 & 30 & \\
\end{tabular}
\end{center}

4T1 = 30 + T2 + T4 -> 4T1 - T2 - 0T3 - T4 = 30\\
4T2 = 60 + T1 + T3 -> -T1 + 4T2 - T3 - 0T4 = 60\\
4T3 = 70 + T2 + T4 -> 0T1 - T2 + 4T3 - T4 = 70\\
4T4 = 40 + T3 + T1 -> -T1 -0T2 - T3 + 4T4 = 40\\

Written out as an augmented matrix Ab this is:
\begin{center}
\begin{tabular}{rrrrr}
4 & -1 & 0 & -1 & 30\\
-1 & 4 & -1 & 0 & 60\\
0 & -1 & 4 & -1 & 70\\
-1 & 0 & -1 & 4 & 40\\
\end{tabular}
\end{center}

After performing row reduction we get:
\begin{center}
\begin{tabular}{rrrrr}
-1 & 0 & -1 & 4 & 40\\
0 & -1 & 4 & -1 & 70\\
0 & 0 & 16 & -8 & 300\\
0 & 0 & 0 & 12 & 270\\
\end{tabular}
\end{center}

And solving gives:\\
x = [20 , 27.5,  30,  22.5]
The following code performs row reduction and solves using back substution:
\begin{verbatim}
import numpy as np;
from numpy import array;
import scipy as sci;
import scipy.linalg as la;
import matplotlib.pyplot as plt;
import operator as op;
from functools import reduce;
lmap = lambda f, *lists: list(map(f,*lists))
#The above imports/defination are assumed in the rest of the code fragments
A = array([[4,-1,0,-1,30],[-1,4,-1,0,60],
           [0,-1,4,-1,70],[-1,0,-1,4,40]],dtype=float)
A[[3,0]] = A[[0,3]]
A[[2,1]] = A[[1,2]]
A[2]-=A[0]
A[2]+=4*A[1]
A[3] += 4*A[0]
A[3] -= A[1]
A[3] += 0.5*A[2]
print(A)
b = A[:,4]
x = np.zeros(4)
for i in range(n-1,-1,-1):
    sum = reduce(op.add,map(lambda j: C[i,j]*x[j],range(i+1,n)),0)
    x[i] = (b[i]-sum)/A[i,i]
print(x)
\end{verbatim}

\item \begin{enumerate}
\item The fact that matrix multiplication is not communicative can be shown simply
by doing out the multiplication of two matrices A and B as AB and BA which
shows that they are not equal.
\begin{center}
\begin{tabular}{llllllll}
x & y & * & a & b & = & ax+cy & bx+dy\\
z & w &  & c & d &  & az+cw & bz+dw\\
\end{tabular}
\end{center}

\begin{center}
\begin{tabular}{llllllll}
a & b & * & x & y & = & ax+bz & ay+bw\\
c & d &  & z & w &  & cx+dz & cy+dw\\
\end{tabular}
\end{center}
\item See a
\item We can see this by comparing the equations for each element in AB and BA.
∃A,B,i,j such that:
(AB[i,j] = sum(k=0\ldots{}n,A[i,k]*B[k,j]) ≠ (BA[i,j] = sum(k=0\ldots{}n,B[i,k]*A[k,j])
\end{enumerate}
\item §1.2 \#33,34
The following code generates the plots for these two problems
\begin{verbatim}
import numpy as np;
# 1.2 33,34
def find_interpolating_polynomial(points):
    degree = len(points)
    x,y = zip(*points)
    A = np.array(lmap(lambda x: lmap(lambda t: x**t, range(degree)), x))
    b = np.array(y)
    x = la.solve(A,b)
    return (A,b,x)
def plot_interpolating_polynomial(points, coefficents, filename):
    x = coefficents
    t = np.linspace(0,10,100)
    f = np.vectorize(lambda t: reduce(op.add,
                                      lmap(lambda i: x[i]*t**i,
                                           range(len(points)))))
    plt.plot(t,f(t));
    plt.plot(*list(zip(*points)), '.');
    plt.savefig(filename);
    plt.clf()
def problem_33():
    points = ((1,12),(2,15),(3,16))
    (A,b,x) = find_interpolating_polynomial(points)
    plot_interpolating_polynomial(points, x, "problem_33.png")
def problem_34():
    points = ((0,0),(2,2.9),(4,14.8),(6,39.6),(8,74.3),(10,119))
    (A,b,x) = find_interpolating_polynomial(points)
    plot_interpolating_polynomial(points, x, "problem_34.png")
problem_33()
problem_34()
\end{verbatim}
\end{enumerate}
\begin{center}
\includegraphics[width=.9\linewidth]{./problem_33.png}
\captionof{figure}{Problem 33 plot}
\end{center}
\begin{center}
\includegraphics[width=.9\linewidth]{./problem_34.png}
\captionof{figure}{Problem 34 plot}
\end{center}

\begin{enumerate}
\item §1.3 \#28
let x = million btu,\\
    y = grams sulfur dioxide, and\\
    z = grams solid particle pollutants\\
A: f(t) = t*(27.6x + 3100y + 250z)\\
B: f(t) = t*(30.2x + 6400y + 360z)\\
\begin{enumerate}
\item 27.6*x1 + 30*x2
\item A(x1) + B(x2), or more explicitly:\\
A(x1) + B(x2) = [27.6*x1,3100*x1,250*x1] + [30.2*x2,6400*x2,360*x2]
\item \begin{center}
\begin{tabular}{rrrllllrrr}
27.6 & 3100 & 250 & * & t1 & t2 & = & 162 & 23610 & 1623\\
30.2 & 6400 & 360\\
\end{tabular}
\end{center}
I'm not really sure how to solve this as a vector equation,
so I'm just going to solve it by hand.

t1 = (162-30.2t2)/27.6 \textasciitilde{}\textasciitilde{} 5.869 - 1.094t2\\
1623 = 360t2 + 360*(5.869 - 1.094t2)\\
1623 = 633.55t2 + 1467.39\\
t2 = 1.8\\
t1 = 3.827\\
So 3.827 tons of A and 1.8 tons of B, with some rounding error.
\end{enumerate}
\item I did this problem using the following code, the function do\(_{\text{problem}}_{\text{5}}\) does
the actual work and prints out the results, the output is shown below.
\begin{verbatim}
def do_problem_5(datafile):
    print_arr = lambda x,y: \
                print("{} =\n{}".format(y,
                                        np.array2string(x,precision = 6,
                                                        suppress_small = True,
                                                        separator=',')))
    np.set_printoptions(precision=6)
    A = loadtxt(datafile)
    (n,m) = A.shape
    (LU,p) = lup_decomp(A)
    (LU_control,p_control) = la.lu_factor(A)
    ## Check that my LU is equal to the actual LU, with a small
    ## tolerence for floating point rouding errors
    assert(np.allclose(LU,LU_control));

    L = np.tril(LU)
    U = np.triu(LU)
    P = np.zeros((n,n))
    for i in range(n):
        L[i,i] = 1
        P[i,p[i]] = 1
    print("Problem 5:")
    print("LUP decomposition")
    print_arr(L,"L")
    print_arr(U,"U")
    print_arr(P,"P")

    print("Solving Ax = b for various values of b")

    b1 = array([2,3,-1,5,7],dtype=float)
    x1 = lup_solve(LU,p,b1)
    x1_control = la.lu_solve((LU_control,p_control),b1)
    assert(np.allclose(x1,x1_control));
    print_arr(b1,"b1")
    print_arr(x1,"x1")

    b2 = array([15,29,8,4,-49],dtype=float)
    x2 = lup_solve(LU,p,b2)
    x2_control = la.lu_solve((LU_control,p_control),b2)
    assert(np.allclose(x2,x2_control));
    print_arr(b2,"b2")
    print_arr(x2,"x2")

    b3 = array([8,-11,3,-8,-32],dtype=float)
    x3 = lup_solve(LU,p,b3)
    x3_control = la.lu_solve((LU_control,p_control),b3 )
    assert(np.allclose(x3,x3_control));
    print_arr(b3,"b3")
    print_arr(x3,"x3")
\end{verbatim}
\begin{verbatim}
Problem 5:
LUP decomposition
L =
[[ 1.      , 0.      , 0.      , 0.      , 0.      ],
 [-0.      , 1.      , 0.      , 0.      , 0.      ],
 [-0.333333, 0.166667, 1.      , 0.      , 0.      ],
 [ 0.666667,-0.833333,-0.5     , 1.      , 0.      ],
 [-0.333333, 0.166667, 0.25    , 0.5     , 1.      ]]
U =
[[ -3.      ,  6.      ,-14.      ,-36.      , -2.      ],
 [  0.      , -6.      ,  4.      ,  6.      ,  0.      ],
 [  0.      ,  0.      , -1.333333, -5.      , -2.666667],
 [  0.      ,  0.      ,  0.      ,  0.5     ,  2.      ],
 [  0.      ,  0.      ,  0.      ,  0.      , -1.      ]]
P =
[[ 0., 0., 0., 0., 1.],
 [ 0., 0., 0., 1., 0.],
 [ 0., 0., 1., 0., 0.],
 [ 0., 1., 0., 0., 0.],
 [ 1., 0., 0., 0., 0.]]
Solving Ax = b for various values of b
b1 =
[ 2., 3.,-1., 5., 7.]
x1 =
[ 19.      ,-18.666667,-47.      , 13.5     , -2.      ]
b2 =
[ 15., 29.,  8.,  4.,-49.]
x2 =
[ 175.      , -37.666667, -57.      ,   1.      ,  30.      ]
b3 =
[  8.,-11.,  3., -8.,-32.]
x3 =
[ 0., 3., 1., 1.,-0.]
\end{verbatim}
\end{enumerate}
\end{document}