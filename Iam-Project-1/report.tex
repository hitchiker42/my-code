\documentclass[twoculumn]{article}
\usepackage{fullpage}
\author{Tucker DiNapoli}
\title{Numerical Analysis of the Korteweg de Vires equation}

\begin{document}
\maketitle
\section{Introduction}
A solver for the one dimensional Korteweg-de-Viers equation was written using C
and parallized using openmpi. The resultant data was plotted with gnuplot in
order to visualize the results. On inspection the plots are consistant with
results for the methods and initial conditions used. The algorithms used are
embarssingly parallel and so parallisation was trivial, and resulted in a
noticable speed up in the code.
\section{KdV Equation}
The Korteweg-de-Vires is a one dimensional partial differential equation, 

\end{document}A program for soving the Korteweg-de-Vires(KdV) partial differential equation
A one dimensional equation used (among other things) to model shallow waves,
in this case u is the height of the water above the sea floor
The equation has the form:
\[\partial_tu+\partial_x^3u-6u\partial_xu=0\]
The solver uses finite-differences in space to discretize the problem
Periodic boundary conditions are used such that the fields satisify:
u(x+Ln) = u(x,y) for all integers n; this implies the grid has a domain of
[0:L] with spacial steps \Delta x=L/mx, where mx is the number of points
the periodic boundary conditions imply that for at a boundary n where
normally n-2,n-1,n,n+1,n+2 would be used to calculate n we use
n-2,n-1,n,0,1
